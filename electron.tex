\documentclass{article}
\usepackage[utf8]{inputenc}
\usepackage{amsmath}
\usepackage{graphicx}
\usepackage{circuitikz}
\usepackage{caption}
\usepackage{subcaption}
\hbadness=99999

\title{Investigation into the charge to mass ratio of an electron}
\author{Di Yan Xu, Rickie Li}
\date{\today}

\begin{document}

\maketitle

\newpage

\section{Introduction}
The purpose of this experiment was to determine the charge to mass ratio of an
electron. This is done by measuring data of electrons which were shot into
an electromagnetic field and measuring the radius of the resulting circle
formed by the electrons. The relation is shown as the following
\begin{align}
    /frac{1}{r} = \sqrt{\frac{e}{2m}}\frac{B}{\sqrt{V}}
\end{align}
Where $r$ is the radius of the electron curvature orbit, $V$ is the voltage at
which the electron is accelerated through and $B$ is the electromagnetic field
which can be broken down to $B_c + B_e$.
$B_c$ being the electromagnetic field generated from the coils and $B_e$ being
the external electromagnetic field from the earth, buildings and other 
equipment/devices. Expanding this we can get our equation to be
\begin{align}
    \frac{1}{r} = \sqrt{\frac{e}{2m}} \frac{1}{\sqrt{V}}\big(B_c + B_e\big)
\end{align}
To determine the charge to mass ratio of en electron we would have to take
datapoints on the radius of the electron orbit with either constant voltage and
varying current, or constant current and varying voltage. In this excercise we
will be using constant voltage and varying current. To determine the ratio we
would first need to rearrange equation (2) and rewrite for $B_c$ to be a function
of $\frac{1}{r}$. Doing so we would get
\begin{align}
    B_c = \sqrt{\frac{2mV}{e}}\frac{1}{r} - B_e
\end{align}

\section{Methods}
Equipment used for the experiment are as follows
\begin{itemize}
    \item[-] Helmholtz coils
    \item[-] Glass bulb
    \item[-] Electron gun
    \item[-] Leybold 0-300V/6.3V power supply
    \item[-] 8V D.C. power supply
    \item[-] Rheostat
    \item[-] 2x Keysight 34461A6 6$\frac{1}{2}$ Digit Multimeter
    \item[-] Self-illuminated scale
    \item[-] Plastic divider
    \item[-] Banana plug wires
\end{itemize}
We set up the circuit as it is shown below INSERT CIRCUIT.  We turn off the lights
in the room and turn on the electron gun for around 30 seconds.  After the 30
seconds we turn on the Helmholtz coils and begin observing and gathering data on the
electron beam's trajectory.  Using the self-illuminated scale, we look through the 
plastic divider to measure the diameter of the beam's trajectory which we later 
convert to radius.  We do this several times with the current going into the coils 
kept constant.  Then we do the same thing several more times with the voltage of 
the 0-300V power supply kept constant.  When doing these we also observe the range 
of the readings on the ammeter and voltmeter to get the error.  After gathering the 
data, we observed how the beam's trajectory is affected when we put our phone, tablet, 
and laptop next to the glass bulb.


\end{document}
