\documentclass{article}
\usepackage[utf8]{inputenc}
\usepackage{amsmath}
\usepackage{graphicx}
\usepackage{circuitikz}
\usepackage{caption}
\usepackage{subcaption}
\hbadness=99999

\title{Investigation into the charge to mass ratio of an electron}
\author{Di Yan Xu, Rickie Li}
\date{\today}

\begin{document}

\maketitle

\newpage

\section{Introduction}
The purpose of this experiment was to determine the charge to mass ratio of an
electron. This is done by shooting electrons into
an electromagnetic field and measuring the radius of the resulting circle
formed by the electrons. The relation is shown as the following
\begin{align}
    \frac{1}{r} = \sqrt{\frac{e}{2m}}\frac{B}{\sqrt{V}}
\end{align}
Where $r$ is the radius of the electron curvature orbit, $V$ is the voltage at
which the electron is accelerated through and $B$ is the electromagnetic field
which can be broken down to $B_c + B_e$.
$B_c$ being the electromagnetic field generated from the coils and $B_e$ being
the external electromagnetic field from the earth, buildings, and other 
equipment/devices. Expanding this we can get our equation to be
\begin{align}
    \frac{1}{r} = \sqrt{\frac{e}{2m}} \frac{1}{\sqrt{V}}\big(B_c + B_e\big)
\end{align}
To determine the charge to mass ratio of an electron we would have to take
data points on the radius of the electron orbit with either constant voltage and
varying current, or constant current and varying voltage. In this exercise we
will be using constant voltage and varying current. To determine the ratio we
would first need to rearrange equation (2) and rewrite for $B_c$ to be a function
of $\frac{1}{r}$. Doing so we would get
\begin{align}
    B_c = \sqrt{\frac{2mV}{e}}\frac{1}{r} - B_e
\end{align}
Using the data we collect, we can run a curve fit and use that to
determine $B_e$ and solve for $\frac{e}{m}$

\newpage

\section{Methods}
Equipment used for the experiment are as follows
\begin{itemize}
    \item[-] Helmholtz coils
    \item[-] Glass bulb
    \item[-] Electron gun
    \item[-] Leybold 0-300V/6.3V power supply
    \item[-] 8V D.C. power supply
    \item[-] Rheostat
    \item[-] 2x Keysight 34461A6 6$\frac{1}{2}$ Digit Multimeter
    \item[-] Self-illuminated scale
    \item[-] Plastic divider
    \item[-] Banana plug wires
\end{itemize}
We set up the circuit as it is shown below INSERT CIRCUIT.  We turn off the lights
in the room and turn on the electron gun for around 30 seconds.  After the 30
seconds we turn on the Helmholtz coils and begin observing and gathering data on the
electron beam's trajectory.  Using the self-illuminated scale, we look through the 
plastic divider to measure the diameter of the beam's trajectory which we later 
convert to radius.  We do this several times with the current going into the coils 
kept constant.  Then we do the same thing several more times with the accelerating 
voltage kept constant.  When doing these we also observe the range 
of the readings on the ammeter and voltmeter to get the error.  After gathering the 
data, we observed how the beam's trajectory is affected when we put our phone, tablet, 
and laptop next to the glass bulb.

\section{Analysis}
\subsubsection*{The problem with parallax}
We used a self-illuminated scale and a plastic reflector to make measurements
of the radius of the electron orbit. If we used standard equipment to measure,
there would be lots of issue with the measurements since the orbit is inside a
round bulb. By using a self-illuminated scale and plastic reflector, we are able
to position the scale in a way that it is measuring the reflected orbit, this
allows us to make more precise measurements and lower the uncertainty.
\subsubsection*{Investigate the anomalous behaviour of the electron trajectory
in the case of low accelerating voltage $V$ and high current in coils $I$}
After running the experiment with a low accelerating voltage and a high current
in coils, we observed that after lowering the voltage to around $100V$, if we go
any lower the electron beam would disappear. When the accelerating voltage was
lower, we can somewhat note that the trajectory of the beam was ever so slightly
shifted. We came to this conclusion because at a lower voltage, we were able to
rotate the bulb to get the beam to form a more solid circle rather than a helix,
after this if we increased the accelerating voltage it can be seen that the
beam is more misaligned.
\subsubsection*{Calculate the extra field $B_e$}
We can calculate the extra field $B_e$ by running curve fit with the data we
collected on equation (3). By doing this we found $B_e = \text{insert vaule here}$

\subsection*{Evaluate the influence of nearby ferromagnetic materials on the
electron trajectory. Is it significant enough to affect the measurement?}
We tried quite a few objects we had on hand, those items being a phone, ipad,
laptop and magnetic laptop keyboard. We noticed that if the object we used has
some type of magnet inside, it would greatly affect the trajectory of the electron,
this can be seen when moving the magnetic side of the keyboard and laptop close
to the bulb. This was also seen when moving the magnetic side of the ipad close
to the bulb. However, when we moved the phone close to the bulb we noticed that
the electron trajectory got a bit blurry but it was still measurable within some
degree of uncertainty. So we can conclude that the influence on the trajectory
would only come from objects with magnets that could influence the electromagnetic
field generated by the coils.

\subsection*{Obtain the value of $\frac{e}{m}$ using the value $B_e$ found above.}




\end{document}
